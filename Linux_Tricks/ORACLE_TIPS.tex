https://support.oracle.com/epmos/faces/DocumentDisplay?_afrLoop=402031519388438&id=1533057.1&displayIndex=2&_afrWindowMode=0&_adf.ctrl-state=qjo2u6w0k_478#aref_section24

AIX HACMP 10g RAC ==> DOC ID 404474.1

archivelog mode
shutdown immediate;
startup mount
alter database archivelog
alter database open

#SID about
SID is System IDentifier
#Show PGA&SGA
SQL> SHOW PGA;
SQL> SHOW SGA;

--
--  Show details of SGA pools.
--
 
SET PAUSE ON
SET PAUSE 'Press Return to Continue'
SET PAGESIZE 60
COLUMN name  FORMAT A30
COLUMN value FORMAT A20
 
SELECT name,
value
FROM   v$parameter
WHERE  name like '%_pool%'
ORDER BY name
/

  oracle sga相关的查询语法

  Posted on 2011/01/27 by adam

  一、当前SGA的总体情况

  select * from
  v$sgainfo;

  二、当前SGA的详细信息

  select * from
  v$sgastat

  空闲的Shared Pool大小:

  select * from
  v$sgastat
  where pool = 'shared pool' and name = 'free memory'

  三、SGA的变化记录

  select * from v$sga_resize_ops

  四、Library Cache的使用情况

  select * from v$librarycache;

  五、Data Dictionary的使用情况

  select sum(gets), sum(getmisses), round(sum(getmisses) * 100 / sum(gets), 2 ) from v$rowcache;

  #下面两个语句与我理想有一点偏差,需要再研究清楚

  select sum(sharable_mem) from v$db_object_cache;

  select sum(sharable_mem) from v$sqlarea;

  –2014-12-24,增加EM没有开启的情况下,查询db_cache_size的优化建议

  select * from V$DB_CACHE_ADVICE;


  1) 检查v$librarycache中sql area的gethitratio是否超过90%,如果未超过90%,应该检查应用代码,提高应用代码的效率。

   

  SQL>select gethitratio from v$librarycache where namespace='SQL AREA';

   

  2) v$librarycache中reloads/pins的比率应该小于1%,如果大于1%,应该增加参数shared_pool_size的值。

   

  select sum(pins),sum(reloads),sum(reloads)/sum(pins) hits from v$librarycache;

   

  reloads/pins>1%有两种可能,一种是library cache空间不足,一种是sql中引用的对象不合法。

   

  select (sum(pins - reloads)) / sum(pins) ``LIB CACHE'' from v$librarycache;

   

  library cache hit ratio > 85%

   

  library cache空间不足,增大share_pool_size(10g)

  SQL>alter system set share_pool_size=’’ spoce=memory/spfile/both

   

  3)shared pool reserved size一般是shared pool size的10%,不能超过50%。V$shared_pool_reserved中的request misses=0或没有持续增长,或者free_memory大于shared pool reserved size的50%,表明shared pool reserved size过大,可以压缩。



#Display active processes.
$ sqlplus /nolog
SQL> connect sys/password@remoteip:1521/instance
SQL> connect sys/password as sysdba;
SQL> DESC V$PROCESS

#Show parameter.
SQL> SHOW PARAMETER db_cache_size;
SQL> SHOW PARAMETER spfile;
SQL> SHOW PARAMETER backgroud_dump;
SQL> SHOW PARAMETER dump_dest;
SQL> SHOW PARAMETER db_name;
SQL> SHOW PARAMETER instance_name;
SQL> SHOW PARAMETER PGA
SQL> SHOW PARAMETER SGA

#Show instance.
SQL> SHOW PARAMETER INSTANCE_NAME
SQL> SHOW PARAMETER SERVICE_NAME
SQL> SELECT INSTANCE_NAME from v$instance;
SQL> SELECT INSTANCE_NAME,STARTUP_TIME,VERSION from v$instance;

$ cat $ORACLE_HOME/rdbms/log/alert_xxx.log
$ ipcs -i
$ ps -ef | grep dbw

#Show datafile.
SQL> SELECT NAME from v$datafile;

#shutdown&start database.
SQL> shutdown immediate;
SQL> startup mount;

#Show recover_file.
SQL> SELECT * from v$recover_file;
SQL> SELECT name from v$datafile where file#=3;

#Control file HeartBeat.
SQL> alter database mount;

#Create orapw
$ orapwd

#Show ADR information.
SQL> SELECT * from v$diag_info;

#ADR Command Interpreter (show alter recorder)
adrci> show alert
adrci> show incident;
adrci> show incident -mode DETAIL -p "incident_id=xxxx";

#Test listener.
$ tnsping SERVICE_NAME
$lsnrctl start
$lsnrctl status

#Set lots of SERVICE_NAME
SQL> ALTER SYSTEM SET SERVICE_NAMES='racdb1,racdb2' scope=both;
SQL> SHOW PARAMETER SERVICE_NAME;

#Show logs.
SQL> SELECT *from v$log;
SQL> SELECT *from v$log;

# 10G AUTOTRACE
SQL> SET AUTOTRACE ON;
SQL> SELECT * from v$version where rownum <2;
SQL> SET AUTOTRACE TRACE EXPLAIN;
SQL> SELECT * from plan_table;
	
# SCN (System Change Name) number.
SQL> SELECT CURRENT_SCN from v$database;
SQL> SHOW PARAMETER CHECKPOINT_TO;

#CheckPoint about. CHECKPOINT T1, CHECKPOINT T2.
|--------+--------+--------|
 	   T1		T2  When CRASH...
	       |---------->| REDO required
>> increace checkpoint
>> check how much Latch point exist.
SQL> SELECT name,gets,misses from v$latch where name='checkpint queue latch';
>> check Latch children points 
SQL> SELECT name,gets,misses from v$latch_children where name='checkpoint queue latch';

SQL> SELECT name,bytes from v$sgastat where upper(name) like '%CHECKPOINT%';
SQL> SELECT name,bytes from v$sgastat where name like 'object queue%';

SQL> SELECT sid,seq#,event from v$session_wait;
SQL> SELECT * v$log;

## Fast-Start Fault Recovery
SQL> SELECT * FROM V$OPTION where Parameter='Fast-Start Fault Recovery';

# SAMPLE CASE: imp 10G data. ROLLBACK
$ imp username/passwd file=xxx.dmp log=xxx.log ignore=y buffer=8192000 feedback=10000 commit=y

SQL> select * from v$rollstat where xacts >0;
SQL> SHOW PARAMETER PARALLEL_ROLLBACK;
SQL> SHOW PARAMETER CPU_COUNT
SQL> SELECT sid,program,event from v$session where PROGRAM like '%P%';

SQL> SELECT usn,undoblockdone,undoblocktotal,cputime from V$FAST_START_TRANSACTIONS;
SQL> SELECT * FROM V$FAST_START_SERVERS;
SQL> SELECT * FROM V$FAST_START_TRANSACTIONS;

#Show dead case
SQL> SELECT distinct KTUXECFL,count(*) from x$ktuxe group by KTUXECFL;
SQL> SELECT ADDR,KTUXEUSN,KTUXESLT,KTUXESQN,KTUXESIZ
2	from x$ktuxe where KTUXEUSN=10 and KTUXESLT=39;
SQL> DECLARE


# TRACE oracle database OPEN processing
SQL> STARTUP MOUNT;
SQL> ALTER SESSION SET SQL_TRACE = TRUE;
SQL> ALTER DATABASE OPEN;

# Use BBED tools edit datafile.
$ bbed parfile=par.bbd
BBED> sum apply 
BBED> verify

>> compile BBED tool in 11g 10g?
$ cd $ORACLE_HOME/rdbms/lib
$ make -f ins_rdbms.mk $ORACLE_HOME/rdbms/lib/bbed


## RMAN use repair bad blocks.

RMAN> backup database format="/opt/data/fullbk.bak' tag='full';
SQL> SELECT count(*)from t;
RMAN> backup validate datafile 2(number show above);
SQL> SELECT * FROM v$database_block_corruption where file#=2;
RMAN> startup mount;
RMAN> blockrecover datafile 2 block 14 from backupset;
	Then done.

# db file scattered read
SQL> create sequence seql;
SQL> create table tl as select seql.nextval,object_id from dba_objects;
SQL> alter table tl modify nextval not null;
SQL> insert into tl select seql.nextval,object_id from dba_objects;
SQL> commit;
SQL> select max(nextval) from tl;
SQL> alter system flush buffer_cache;
SQL> select object_id from tl where nextval>1000 and nextval<=100000;
SQL> 

# About instance mem and instruction.
SQL> 
select COMPONENT,CURRENT_SIZE,MIN_SIZE,MAX_SIZE from v;

select name,value from v$pgastat
where name in ('maximum PGA allocated','total PGA allocated');

select program from v$process order by program;

# Show platform.
SQL> select platform_name from v$database;

# Show controlfile
SQL> select name from v$controlfile;

# Show exist group...and list.
select group#,bytes,members from v$log;
select group#,member from v$logfile;


# How to check the maximum number of allowed connections to an Oracle database?

SELECT
  'Currently, ' 
  || (SELECT COUNT(*) FROM V$SESSION)
  || ' out of ' 
  || VP.VALUE 
  || ' connections are used.' AS USAGE_MESSAGE
FROM 
  V$PARAMETER VP
WHERE VP.NAME = 'sessions'



> The number of sessions the database was configured to allow

SELECT name, value 
  FROM v$parameter
   WHERE name = 'sessions'

> The number of sessions currently active

   SELECT COUNT(*)
     FROM v$session


# Change processes , sessions, transactions !
processes=x
sessions=x*1.1+5
transactions=sessions*1.1
 
E.g.
processes=500
sessions=555
transactions=610
 
select name, value from v$parameter where name in ('sessions','processes','transactions');


sql> alter system set processes=500 scope=both sid='*';
sql> alter system set sessions=555 scope=both sid='*';
sql> alter system set transactions=610 scope=both sid='*';


https://forums.oracle.com/forums/thread.jspa?threadID=898395
http://docs.oracle.com/cd/B19306_01/server.102/b14237/initparams191.htm


http://avdeo.com/2012/06/28/sessions-and-processes-parameters-oracle-11g/
http://webhelp.esri.com/arcgisserver/9.3/java/index.htm#geodatabases/oracle_1010194088.htm


sql> alter system set processes=500 scope=spfile sid='*';
sql> alter system set sessions=555 scope=spfile sid='*';
sql> alter system set transactions=610 scope=spfile sid='*';


# Show NLS_LANG setting
SELECT value$ FROM sys.props$ WHERE name = 'NLS_CHARACTERSET' ;

SELECT * from NLS_SESSION_PARAMETERS;
# http://www.oracle.com/technetwork/database/globalization/nls-lang-099431.html

# How to find the NLS_LANG to set for a database?
select DECODE(parameter, 'NLS_CHARACTERSET', 'CHARACTER SET',
'NLS_LANGUAGE', 'LANGUAGE',
'NLS_TERRITORY', 'TERRITORY') name,
value from v$nls_parameters
WHERE parameter IN ( 'NLS_CHARACTERSET', 'NLS_LANGUAGE', 'NLS_TERRITORY')

# Show user role
SQL> SELECT * FROM user_role_privs;


# Check free/used space per tablespace

Example query to check free and used space per tablespace:

SELECT /* + RULE */  df.tablespace_name "Tablespace",
       df.bytes / (1024 * 1024) "Size (MB)",
       SUM(fs.bytes) / (1024 * 1024) "Free (MB)",
       Nvl(Round(SUM(fs.bytes) * 100 / df.bytes),1) "% Free",
       Round((df.bytes - SUM(fs.bytes)) * 100 / df.bytes) "% Used"
  FROM dba_free_space fs,
       (SELECT tablespace_name,SUM(bytes) bytes
          FROM dba_data_files
         GROUP BY tablespace_name) df
 WHERE fs.tablespace_name (+)  = df.tablespace_name
 GROUP BY df.tablespace_name,df.bytes
UNION ALL
SELECT /* + RULE */ df.tablespace_name tspace,
       fs.bytes / (1024 * 1024),
       SUM(df.bytes_free) / (1024 * 1024),
       Nvl(Round((SUM(fs.bytes) - df.bytes_used) * 100 / fs.bytes), 1),
       Round((SUM(fs.bytes) - df.bytes_free) * 100 / fs.bytes)
  FROM dba_temp_files fs,
       (SELECT tablespace_name,bytes_free,bytes_used
          FROM v$temp_space_header
         GROUP BY tablespace_name,bytes_free,bytes_used) df
 WHERE fs.tablespace_name (+)  = df.tablespace_name
 GROUP BY df.tablespace_name,fs.bytes,df.bytes_free,df.bytes_used
 ORDER BY 4 DESC;

 # Sometimes Oracle takes forever to shutdown with the ``immediate'' option. As workaround to this problem, shutdown using these commands:

 alter system checkpoint;
 shutdown abort
 startup restrict
 shutdown immediate

 Note that if your database is in ARCHIVELOG mode, one can still use archived log files to roll forward from an off-line backup. If you cannot take your database down for a cold (off-line) backup at a convenient time, switch your database into ARCHIVELOG mode and perform hot (on-line) backups. 

 # List all archivelog backups for the past 24 hours:

 RMAN> LIST BACKUP OF ARCHIVELOG FROM TIME 'sysdate-1';

 # Setting RMAN recovery window to 14 days &=& > any backup older than 14 days become obsolete. Incorrect

  http://oradbatips.blogspot.com/2006/11/tip-4-rman-recovery-window.html

 #  SQL>  SELECT INSTANCE_NAME, DATABASE_STATUS, INSTANCE_ROLE from v$instance; 

# 20130605

## Troubshooting

when the "dbstart" command haven't happen anything 
check /etc/orata file, replace "N" with "Y"

## tnsnames.ora Network Configuration File define the Remote login port and 
so on..
CONN SYSTEM/password@192.168.1.111:1521/oradb   ( port 1521 is ..)

## sqlnet.ora File define OS authentication and Oracle authentication.
SQLNET.AUTHENTICATION_SERVICES=(False)  -> mean Oracle authentication
SQLNET.AUTHENTICATION_SERVICES=(All)    -> mean Linux OS authentication

## Recovery Oracle orapw$ORACLE_SID file.
loggin as Sys or System on the EM,the error i get is ``Your username and/or 
password are invalid.''

The value of Remote_login_passwordfile=Exclusive parameter should pfile/spfi
le.
Then just create a password file using ``orapwd'' utility and then try to lo
gged in. it will solve your problem.
=> $ orapwd file='$ORACLE_HOME/dbs/orapw$ORACLE_SID' password=pwd entries=10
 force=y

SELECT * FROM NLS_DATABASE_PARAM\begin{table}

# Lost Oracle SYS and SYSTEM password?

$ sqlplus "/ as sysdba"
SQL> show user
SQL> passw system
SQL> passw sys

# SPFILE 错误导致数据库无法启动(ORA-01565) 
http://blog.csdn.net/robinson_0612/article/details/5774795

# uninstall Oracle database 10g 
# AS root
root # export ORACLE_HOME=/u01/app/oracle/product/10.2.0/db_1

# How to drop DATABASE INSTANCE?

https://blogs.oracle.com/AlejandroVargas/entry/10g_drop_database_command

The steps are as follows:

    shutdown abort;
    startup mount exclusive restrict;
    drop database;

This is a sample session using this feature:

# Drop tablespace 
http://psoug.org/snippet/TABLESPACE-Dropping-Tablespaces_851.htm

startup mount
alter database datafile '/path/to/filename' offline drop;
alter database open;
drop tablespace tbsp_name including contents cascade constraints;


# How to patch 10.2.0.4 from 10.2.0.1 in RHEL5
http://www.oracleflash.com/29/Upgrade-Oracle-10g-Release-2-from-10201-to-10204.html
# How to patch 10.2.0.1 to 10.2.0.4 in Oracle 10g RAC?
http://khalidali-oracledba.blogspot.com/2012/03/rac-upgradation-10201-to-10204.html

# In order to see the database default tablespace issue,

SQL> SELECT PROPERTY_VALUE FROM DATABASE_PROPERTIES WHERE property_name = 'DEFAULT_PERMANENT_TABLESPACE';

PROPERTY_VALUE
--------------------------------------------------------------------------------
USERS

# In order to see a particular user default tablespace use,
SQL> SELECT DEFAULT_TABLESPACE FROM DBA_USERS WHERE USERNAME='WZOA';

DEFAULT_TABLESPACE
------------------------------
USER_TBS

# Delete an Instance from an Oracle RAC Database
http://www.oracle-base.com/articles/rac/delete-an-instance-from-an-oracle-rac-database.php

dbca -silent -deleteInstance -nodeList ol5-112-rac2 -gdbName RAC -instanceName RAC2 -sysDBAUserName sys -sysDBAPassword myPassword


#
#---------------------------------------------------------------------------------------------------

# RMAN restore 

1, Here is a restore example:

RMAN> run {
 allocate channel dev1 type disk;
 restore (archivelog low logseq 78311 high logseq 78340 thread 1 all);
 release channel dev1;
}

2,Example RMAN restore:

rman target sys/*** nocatalog 
run {
  allocate channel t1 type disk;
  # set until time 'Aug 07 2000 :51';
  restore tablespace users; 
  recover tablespace users; 
  release channel t1; 
}
#---------------------------------------------------------------------------------------------------

#---------------------------------------------------------------------------------------------------
#ORA-01157: cannot identify/lock data file 7 - see DBWR trace file
#ORA-01110: data file 7: '/data/WZG_OA.dbf'

SQL> alter database datafile '/data/WZG_OA.dbf' offline drop;
SQL> 

 既然出现报错的几个dbf文件已经不用,则解决办法相对简单,只要将对应的数据文件删除,并继续删除对应新增的表空间即可。操作过程如下:

 SQL> shutdown immediate;

 SQL> startup mount;

 SQL> select file#,name,status from v$datafile;

 SQL> alter database datafile '/tmp/test.dbf' offline drop;      //此处若不加drop会报错

 再次查看v$datafile表会发现对应的几个dbf文件状态由ONLINE变为RECOVER

 SQL> select * from v$tablespace;

 SQL> drop tablespace test including contents cascade constraints;


 删除完毕,再次执行startup成功。
#---------------------------------------------------------------------------------------------------

#---------------------------------------------------------------------------------------------------
# about expdp/impdp

Expdp@impdp用法
維基教科書,自由的教學讀本
跳轉到: 導覽、 搜尋

一、創建導出數據存放目錄

如:mkdir /u01/dump

二、創建directory邏輯目錄

CREATE OR REPLACE DIRECTORY DATA_DUMP_DIR AS '/u01/dump';

三、導出數據

1)按用戶導

expdp mctpsa/mctpsa@ipap schemas=mctpsa dumpfile=expdp.dmp DIRECTORY=DATA_DUMP_DIR;

2)並行進程parallel

expdp mctpsa/mctpsa@ipap directory=DATA_DUMP_DIR dumpfile=mctpsa3.dmp parallel=40 job_name=mctpsa3

3)按表名導

expdp mctpsa/mctpsa@ipap TABLES=sa_user,sa_dept dumpfile=expdp.dmp DIRECTORY=DATA_DUMP_DIR;

4)按查詢條件導

expdp mctpsa/mctpsa@ipap directory=DATA_DUMP_DIR dumpfile=expdp.dmp Tables=sa_user query='WHERE id=20';

5)按表空間導

expdp system/manager DIRECTORY=DATA_DUMP_DIR DUMPFILE=tablespace.dmp TABLESPACES=mctp,mctpsa;

6)導整個數據庫

expdp system/manager DIRECTORY=DATA_DUMP_DIR DUMPFILE=full.dmp FULL=y;


四、還原數據

1)導到指定用戶下

impdp mctpsa/mctpsa DIRECTORY=DATA_DUMP_DIR DUMPFILE=expdp.dmp SCHEMAS=mctpsa;

2)改變表的owner

impdp system/manager DIRECTORY=DATA_DUMP_DIR DUMPFILE=expdp.dmp TABLES=mctpsa.dept REMAP_SCHEMA=mctpsa:system;

3)導入表空間

impdp system/manager DIRECTORY=DATA_DUMP_DIR DUMPFILE=tablespace.dmp TABLESPACES=example;

4)導入數據庫

impdb system/manager DIRECTORY=DATA_DUMP_DIR DUMPFILE=full.dmp FULL=y;

5)追加數據

impdp system/manager DIRECTORY=dpdata1 DUMPFILE=expdp.dmp SCHEMAS=system TABLE_EXISTS_ACTION=append;
#---------------------------------------------------------------------------------------------------


# query database size.

select sum(bytes)/1024/1024/1024 from dba_segments;

select
( select sum(bytes)/1024/1024 data_size from dba_data_files ) +
( select nvl(sum(bytes),0)/1024/1024 temp_size from dba_temp_files ) +
( select sum(bytes)/1024/1024 redo_size from sys.v_$log ) +
( select sum(BLOCK_SIZE*FILE_SIZE_BLKS)/1024/1024 controlfile_size from v$controlfile) "Size in MB"
from
dual;

# 如何把數據庫導入一個新環境(新的數據庫)?

1) 獲取原數據庫用戶及其表信息。
select * from all_users;

SQL> select * from all_users;

USERNAME			  USER_ID CREATED
------------------------------ ---------- ----------
QILIPORT2006			       60 22-11?-10
MLSTOCK 			       58 22-11?-10
RATION				       57 22-11?-10
QILIPORT			       59 22-11?-10
SCOTT				       54 29-10?-05
MGMT_VIEW			       53 29-10?-05
MDDATA				       50 29-10?-05
SYSMAN				       51 29-10?-05
MDSYS				       46 29-10?-05
SI_INFORMTN_SCHEMA		       45 29-10?-05
ORDPLUGINS			       44 29-10?-05

USERNAME			  USER_ID CREATED
------------------------------ ---------- ----------
ORDSYS				       43 29-10?-05
OLAPSYS 			       47 29-10?-05
ANONYMOUS			       39 29-10?-05
XDB				       38 29-10?-05
CTXSYS				       36 29-10?-05
EXFSYS				       34 29-10?-05
WMSYS				       25 29-10?-05
DBSNMP				       24 29-10?-05
TSMSYS				       21 29-10?-05
DMSYS				       35 29-10?-05
DIP				       19 29-10?-05

USERNAME			  USER_ID CREATED
------------------------------ ---------- ----------
OUTLN				       11 29-10?-05
SYSTEM					5 29-10?-05
SYS					0 29-10?-05
以上可知哪些用戶需要在新環境創建。

SQL> select * from user_role_privs;  

USERNAME               GRANTED_ROLE           ADM DEF OS_
------------------------------ ------------------------------ --- --- ---
QILIPORT               CONNECT                NO  YES NO
QILIPORT               RESOURCE               NO  YES NO
以上獲取相應用戶的權限。

收集需要的用戶信息供新環境創建,比如:
conn jiliport
select * from user_users;

SQL> select * from user_users;

USERNAME			  USER_ID ACCOUNT_STATUS
------------------------------ ---------- --------------------------------
LOCK_DATE  EXPIRY_DAT DEFAULT_TABLESPACE
---------- ---------- ------------------------------
TEMPORARY_TABLESPACE	       CREATED	  INITIAL_RSRC_CONSUMER_GROUP
------------------------------ ---------- ------------------------------
EXTERNAL_NAME
--------------------------------------------------------------------------------
QILIPORT			       59 OPEN
		      USERS
TEMP			       22-11?-10  DEFAULT_CONSUMER_GROUP
以上可知qiliport用戶使用USERS和TEMP表空間。


2) 在新環境數據數據庫中創建相應的用戶和表空間。
比如:



Oracle數據庫如何查看當前用戶角色權限及默認表空間:
http://www.itkee.com/database/detail-965.html


查看數據文件放置的路徑
 SELECT TABLESPACE_NAME,BYTES/1024/1024 MB,FILE_NAME FROM DBA_DATA_FILES;

	
单击此项可添加到收藏夹		Export/Import DataPump Parameter VERSION - Compatibility of Data Pump Between Different Oracle Versions [Video] (文档 ID 553337.1)	转到底部转到底部	
修改时间:2012-11-22类型:BULLETIN	
为此文档评级	通过电子邮件发送此文档的链接	在新窗口中打开文档	可打印页



https://support.oracle.com/epmos/faces/DocumentDisplay?_afrLoop=402294074981002&id=553337.1&_afrWindowMode=0&_adf.ctrl-state=qjo2u6w0k_583
. Import data into a target database with a higher compatibility level.
Sample configuration 1:

Source database: 10.1.0.5.0 with COMPATIBLE=10.1.0
Target database: 10.2.0.4.0 with COMPATIBLE=10.2.0

Solution:

    Start the export job using the 10.1.0.5.0 Export Data Pump client that connects to the 10.1.0.5.0 source database.
    Transfer the dumpfile set to the server where the target database is located.
    Start the import of the data using the 10.2.0.4.0 Import Data Pump client that connects to the 10.2.0.4.0 target database.

-- Step 1: export from 10.1.0.5 source database
-- with 10.1.0.5 Export Data Pump client:

% expdp system/manager DIRECTORY=my_dir DUMPFILE=expdp_s.dmp \
LOGFILE=expdp_s.log SCHEMAS=scott

-- Step 2: transfer dumpfile to target server

-- Step 3: import into 10.2.0.4 target database 
-- with 10.2.0.4 Import Data Pump client:

% impdp system/manager DIRECTORY=my_dir DUMPFILE=expdp_s.dmp \
LOGFILE=impdp_s.log SCHEMAS=scott


0002
########

-- Step 1: create database link on local 10.2.0.4 target
-- database that connects to remote 10.1.0.5 source database:

CONNECT system/manager
CREATE DATABASE LINK my_dblink 
   CONNECT TO system IDENTIFIED BY manager
   USING 'm10105wa.oracle.com';

-- Step 2: import with 10.2.0.4 Import Data Pump client
-- from 10.1.0.5 source database over database link: 

% impdp system/manager LOGFILE=my_dir:impdp_s.log \
SCHEMAS=scott NETWORK_LINK=my_dblink

# SHMMAX size about
http://www.wallcopper.com/database/657.html
SHMMAX Available physical memory Defines the maximum allowable size
of one shared memory segment. The SHMMAX setting should be large enough
to hold the entire SGA in one shared memory segment. A low setting can
cause creation of multiple shared memory segments which may lead to
performance degradation.

Oracle recommends half the RAM.
Edit /etc/sysctl.conf

kernel.shmmax = 7730941132
SQL> alter system set sga_max_size=6G scope=spfile;

## Upgrade Oracle 10g Release 2 from 10.2.0.1 to 10.2.0.5
# http://p-mayuranathan.blogspot.com/2012/07/patch-upgradation-from-10201-to-10205.html

# Patch download link
https://updates.oracle.com/Orion/PatchDetails/switch_to_simple?plat_lang=233P&patch_file=&file_id=&password_required=&password_required_readme=&merged_trans=&aru=12791168&patch_num=8202632&patch_num_id=1253975&default_release=80102050&default_plat_lang=233P&default_compatible_with=&patch_password=&orderby=&direction=&no_header=0&sortcolpressed=&tab_number=


#  How to check free space in ASM 
select name, state, total_mb, free_mb from v$asm_diskgroup;

SELECT name, type, ceil (total_mb/1024) TOTAL_GB , ceil (free_mb/1024) FREE_GB, required_mirror_free_mb,ceil ((usable_file_mb)/1024) FROM V$ASM_DISKGROUP;


# 11g account expired
Disable Oracle's password expiry
http://www.odi.ch/weblog/posting.php?posting=520

# add datafile to tablespace

alter tablespace OA add datafile '/oratest/OA02.dbf' size 10G;

# http://linuxtechres.blogspot.com/2009/11/how-to-clean-up-disk-space-for-oracle.html

# How to see the oldest flashback available?
Using the following query one can see the flashback data available.

SELECT to_char(sysdate,'YYYY-MM-DD HH24:MI') current_time, to_char(f.oldest_flashback_time, 'YYYY-MM-DD HH24:MI') OLDEST_FLASHBACK_TIME,
(sysdate – f.oldest_flashback_time)*24*60 HIST_MIN FROM v$database d, V$FLASHBACK_DATABASE_LOG f;

CURRENT_TIME OLDEST_FLASHBACK HIST_MIN


#  recover the oracle RMAN tablespace if it is dropped
-----------------------------------------------------------------

Step 1:
Restore the control file that was taken before the tablespace was dropped
 

Step 2:

Restore and recover the db
---------------------------------------------------------------------------
RMAN> run
2> {
3> set UNTIL TIME "to_date('YYYY-MM-DD HH24:MI:SS')";
4> restore database;
5> restore archivelog;
6> }
RMAN> recover database until cancel using backup controlfile;
RMAN> alter database open resetlogs;
--------------------------------------------------------------------------

Check if the tablespace is visible after recovery of database :
select tablespace_name from dba_tablespaces where tablespace_name = '<table space name>';
 

After recovery is complete, you can try with this options
RMAN> alter tablespace <tablespacename> offline;
RMAN> restore tablespace <tablespacename>;
RMAN> recover tablespace  <tablespacename>;
RMAN> alter tablespace online;
=========================================================
-----------------------------------------------------------------

# http://labite.wordpress.com/2012/08/31/tuning-linux-server-for-oracle-database/


# SCAN IP 
 Define the SCAN in your corporate DNS (Domain Name Service)

 If you choose Option 1, you must ask your network administrator to create a single name that resolves to 3 IP addresses using a round-robin algorithm. Three IP addresses are recommended considering load balancing and high availability requirements regardless of the number of servers in the cluster. The IP addresses must be on the same subnet as your public network in the cluster. The name must be 15 characters or less in length, not including the domain, and must be resolvable without the domain suffix (for example: “krac-scan’ must be resolvable as opposed to “krac-scan.india.com”). The IPs must not be assigned to a network interface (on the cluster), since Oracle Clusterware will take care of it.

 kracnode-scan     192.168.1.72
                   192.168.1.70
                                     192.168.1.71


# emctl dbconsole about
http://blog.mclaughlinsoftware.com/oracle-architecture-configuration/changing-windows-hostname-and-oracle-enterprise-manager/
1, emca -deconfig dbcontrol db -repos drop
2, emca -config dbcontrol db -repos create

# rlwrap 
http://bereanstechnology.com/index.php?q=node/42

#  Advanced reciplication VS stream VS Goldgate
http://blog.devart.com/ten-ways-to-synchronize-oracle-table-data.html

# delete log
rlwrap adrci
show home 
set home xxx
 purge -age 14


# To disable FRA you can use:
  ALTER SYSTEM SET DB_RECOVERY_FILE_DEST = '' scope=both;

## ORA-00205: error in identifying control file, check alert log for more info

#  http://www.oracledistilled.com/oracle-database/recover-from-a-corrupt-or-missing-control-file/


# Windows archivelog automatic clean how to.

 CONFIGURE RETENTION POLICY TO RECOVERY WINDOW OF 3 DAYS;
 CONFIGURE CONTROLFILE AUTOBACKUP ON;


# How to re-synchronize the streams replicated objects online 
http://eric-oracle.blogspot.com/2008/12/how-to-re-synchronize-streams.html
http://kubilaykara.blogspot.com/2010/10/oracle-11g-streams-synchronous-capture.html
http://krish-dba.blogspot.com/2009/01/re-synchronizingrefresh-table-in.html


# Oracle Stream
http://wedostreams.blogspot.com/2009/01/oracle-streams-101.html#streams101p5
http://orachat.com/10g-streams-configuration/
http://www.askmaclean.com/archives/tag/streams
http://anargodjaev.wordpress.com/2014/01/03/step-by-step-example-on-setting-up-streams-oracle-11g/
#http://blog.itpub.net/25198367/viewspace-715070
http://www.orafaq.com/wiki/Oracle_Streams#Components_of_Oracle_Streams
http://wenku.baidu.com/view/2fcbf64d2e3f5727a5e962fa.html
http://www.orafaq.com/wiki/Oracle_Streams#Components_of_Oracle_Streams

# awr report 
http://www.pafumi.net/AWR%20Reports.html#SGA_Target_Advisory

sqlplus / as sysdba
@?/rdbms/admin/awrrpt.sql
7days
begin to end
filename

Cache Sizes

  Begin     End     
    
  Buffer Cache:     1,520M  1,344M  Std Block Size:     8K
  Shared Pool Size:     1,120M  1,296M  Log Buffer:     8,632K

  Elasped Time: It represents the snapshot window or the time between the two snapshots.
  DB TIME: Represents the activity on the database.

  If DB TIME is Greater than Elapsed Time then it means that database has high workload.


# How do you get the datafile sizes from the oracle? 
COL FILE_NAME FOR A70
COL TABLESPACE_NAME FOR A30
CLEAR BREAKS
CLEAR COMPUTES
COMPUTE SUM OF SIZE_IN_MB ON REPORT
BREAK ON REPORT
SELECT TABLESPACE_NAME,FILE_NAME,AUTOEXTENSIBLE,
INCREMENT_BY,MAXBYTES/1024/1024 "MAX IN MB",
BYTES/1024/1024 "SIZE_IN_MB"
FROM DBA_DATA_FILES ORDER BY TABLESPACE_NAME;

# Processing object type EXPORT/TABLE/INDEX/INDEX [message #505129 is a reply to message #505045]
 I solved my problem.I believe.It happened due to space problem.I did free up some space (around 50 gb). I ran the import again and it took 4 hrs to complete. Replying thinking it might help others.


# Add space to database

SQL> ALTER TABLESPACE ts1 ADD DATAFILE '/path/to/file/name' SIZE 100M;

OR change it
SQL> ALTER DATABASE DATAFILE '/path/to/data/file/name' RESIZE 200M;
