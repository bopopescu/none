DRBD (Distributed Replicated Block Device) 是 Linux 平台上的分散式储存系统。其中包含了核心模组,数个使用者空间管理程式及 shell scripts,通常用于高可用性(high availability, HA)丛集。DRBD 类似磁盘阵列的RAID 1(镜像),只不过 RAID 1 是在同一台电脑内,而 DRBD 是透过网络。
Overview of DRBD concept

DRBD 是以 GPL2 授权散布的自由软件。

drbd 工作原理
DRBD是一种块设备,可以被用于高可用(HA)之中.它类似于一个网络RAID-1功能.当你将数据写入本地 文件系统时,数据还将会被发送到网络中另一台主机上.以相同的形式记录在一个文件系统中。 本地(主节点)与远程主机(备节点)的数据可以保证实时同步.当本地系统出现故障时,远程主机上还会 保留有一份相同的数据,可以继续使用.在高可用(HA)中使用DRBD功能,可以代替使用一个共享盘阵.
因为数据同时存在于本地主机和远程主机上,切换时,远程主机只要使用它上面的那份备份数据,
就可以继续进行服务了。
两台机器的环境如下:
hostname:srv5.localdomain
192.168.8.5
hostname:srv6.localdomain
192.168.8.6

准备环境

在两台机器上各新加一块磁盘
fdisk -l 列出所有的磁盘和分区的情况
在实验中可以看到新加的磁盘还没有分区
Disk /dev/sdb: 1073 MB, 1073741824 bytes
255 heads, 63 sectors/track, 130 cylinders
Units = cylinders of 16065 * 512 = 8225280 bytes

对新加的磁盘分区
[root@srv5 ~]# fdisk /dev/sdb

Command (m for help): n
Command action
   e   extended
   p   primary partition (1-4)
e
Partition number (1-4): 1
First cylinder (1-130, default 1):
Using default value 1
Last cylinder or +size or +sizeM or +sizeK (1-130, default 130):
Using default value 130

Command (m for help): n
Command action
   l   logical (5 or over)
   p   primary partition (1-4)
l
First cylinder (1-130, default 1):
Using default value 1
Last cylinder or +size or +sizeM or +sizeK (1-130, default 130):
Using default value 130

Command (m for help): w
The partition table has been altered!

Calling ioctl() to re-read partition table.
Syncing disks.
加载磁盘信息
partprobe /dev/sdb
cat /proc/partions

安装和配置DRBD

1,两台机器上分别安装drbd
yum -y install kmod-drbd83 drbd83
检查是否安装成功
[root@srv5 yum.repos.d]# modprobe -l | grep -i drbd
/lib/modules/2.6.18-53.el5/weak-updates/drbd83/drbd.ko
安装完成后再/sbin 目录下有drbd的命令文件, 在/etc/init.d/目录下有drbd启动脚本
[root@srv5 yum.repos.d]# ls /sbin/drbd*
/sbin/drbdadm  /sbin/drbdmeta  /sbin/drbdsetup

2,配置drbd
2.1,在两台机器的hosts文件中添加如下的内容:
192.168.8.5 srv5.localdomain
192.168.8.6 srv6.localdomain

2.2,DRBD运行的时候要读取/etc/drbd.conf文件,
将文件的内容保存为如下内容:
include "drbd.d/global_common.conf";
include "drbd.d/*.res";

修改global_common.conf文件内容如下:
[html] view plaincopy在CODE上查看代码片派生到我的代码片

    global {  
        usage-count no;  
    }  
    common {  
        protocol C;  
        startup {  
            wfc-timeout 15;    
            degr-wfc-timeout 15;    
            outdated-wfc-timeout 15;    
        }  
        disk {  
            on-io-error detach;    
            fencing resource-only;    
        }  
        net {  
            cram-hmac-alg sha1;    
            shared-secret "123456";     
        }  
        syncer {  
            rate 100M;    
        }  
    }  

创建一个xserver.res文件内容如下:
[html] view plaincopy在CODE上查看代码片派生到我的代码片

    resource xserver {     
        meta-disk internal;     
        device /dev/drbd0; #device指定的参数最后必须有一个数字,用于global的minor-count,  
        #否则会报错。device指定drbd应用层设备。   
        on srv5.localdomain{    #注意:drbd配置文件中,机器名大小写敏感!  
            address 192.168.8.5:7789;     
            disk /dev/sdb5;      
        }     
        on srv6.localdomain {     
            address 192.168.8.6:7789;     
            disk /dev/sdb5;    
        }     
    }  



2.3,假设上面的配置是在92.168.8.5上做的,拷贝上面配置好的3个文件到92.168.8.6上
scp /etc/drbd.conf 192.168.8.6:/etc/drbd.conf
scp /etc/drbd.d/* 192.168.8.6:/etc/drbd.d/

3,在两台机器上创建drbd元数据信息:
[root@srv5 ~]# drbdadm create-md all
Writing meta data...
initializing activity log
NOT initialized bitmap
New drbd meta data block successfully created.

4,启动服务,
4.1,在主节点92.168.8.5上
service drbd start
在备份节点92.168.8.6上
service drbd start
在两台机器上用下面的命令drbd-overview或者cat /proc/drbd查看,发现都是Secondary 备份状态
[root@srv5 ~]# drbd-overview
  0:xserver  Connected Secondary/Secondary Inconsistent/Inconsistent C r-----
[root@srv6 ~]# cat /proc/drbd
version: 8.3.15 (api:88/proto:86-97)
GIT-hash: 0ce4d235fc02b5c53c1c52c53433d11a694eab8c build by mockbuild@builder17.centos.org, 2013-03-27 16:04:08
 0: cs:Connected ro:Secondary/Secondary ds:Inconsistent/Inconsistent C r-----
    ns:0 nr:0 dw:0 dr:0 al:0 bm:0 lo:0 pe:0 ua:0 ap:0 ep:1 wo:b oos:1044092

4.2,在主节点92.168.8.5上执行下面的命令让其成为主节点
drbdadm -- --overwrite-data-of-peer primary all
然后再看状态:
[root@srv5 ~]# drbd-overview
  0:xserver  SyncSource Primary/Secondary UpToDate/Inconsistent C r---n-
        [=============>......] sync'ed: 70.6% (310268/1044092)K
[root@srv5 ~]# drbd-overview
  0:xserver  SyncSource Primary/Secondary UpToDate/Inconsistent C r---n-
        [================>...] sync'ed: 89.1% (117372/1044092)K
[root@srv5 ~]# drbd-overview
  0:xserver  Connected Primary/Secondary UpToDate/UpToDate C r-----

4.3,将/dev/drbd0格式化并挂载
在主节点192.168.8.5上执行下面的命令
mkfs.ext3 /dev/drbd0
mkdir /xserver-storage
mount /dev/drbd0 /xserver-storage
4.4,测试同步
在主节点192.168.8.5上执行下面的命令
cd /xserver-storage
echo "a file created in server5" > testfile
在备份节点192.168.8.6上执行下面的命令
mkdir /xserver-storage
mount /dev/drbd0 /xserver-storage
#mount 会出错,因为mount只能在Primary一端使用
在主节点192.168.8.5上执行下面的命令变成备份节点
umount /xserver-storage
drbdadm secondary all
在备份节点192.168.8.6上执行下面的命令变为主节点,可以看到192.168.8.5同步过来的内容
drbdadm primary all
mount /dev/drbd0 /xserver-storage
less /xserver-storage/testfile
可以查看到文件的内容为"a file created in server5"
在192.168.8.6新建一个文件
echo "a file created in server6" > testfile2
再将192.168.8.5变为主节点后mount可以看到testfile2的内容也同步了。
